%\iffalse
\let\negmedspace\undefined
\let\negthickspace\undefined
\documentclass[journal,12pt,twocolumn]{IEEEtran}
\usepackage{cite}
\usepackage{amsmath,amssymb,amsfonts,amsthm}
\usepackage{algorithmic}
\usepackage{graphicx}
\usepackage{textcomp}
\usepackage{xcolor}
\usepackage{txfonts}
\usepackage{listings}
\usepackage{enumitem}
\usepackage{mathtools}
\usepackage{gensymb}
\usepackage{comment}
\usepackage[breaklinks=true]{hyperref}
\usepackage{tkz-euclide} 
\usepackage{listings}
\usepackage{gvv}                                        
\def\inputGnumericTable{}                                 
\usepackage[latin1]{inputenc}                                
\usepackage{color}                                            
\usepackage{array}                                            
\usepackage{longtable}                                       
\usepackage{calc}                                             
\usepackage{multirow}                                         
\usepackage{hhline}                                           
\usepackage{ifthen}                                           
\usepackage{lscape}

\newtheorem{theorem}{Theorem}[section]
\newtheorem{problem}{Problem}
\newtheorem{proposition}{Proposition}[section]
\newtheorem{lemma}{Lemma}[section]
\newtheorem{corollary}[theorem]{Corollary}
\newtheorem{example}{Example}[section]
\newtheorem{definition}[problem]{Definition}
\newcommand{\BEQA}{\begin{eqnarray}}
\newcommand{\EEQA}{\end{eqnarray}}
\newcommand{\define}{\stackrel{\triangle}{=}}
\theoremstyle{remark}
\newtheorem{rem}{Remark}

\begin{document}
\bibliographystyle{IEEEtran}
\vspace{3cm}
\title{\textbf{IN-2023}}
\author{EE23BTECH1153-R.Rahul$^{*}$% <-this % stops a space
}
\maketitle
\newpage
\bigskip

\textbf{QUESTION:}\\
61. In the diagram shown, the frequency of the sinusoidal source voltage $V_s$ is 50 Hz.The load voltage is 230 V (RMS), and the load impedance is $\frac{230}{\sqrt{2}}$+$j\frac{230}{\sqrt{2}}$ $\Omega$. The value of attenuator $A_1$=$\frac{1}{50\sqrt{2}}$.The multiplier output voltage $V_o=\frac{V_xV_y}{1V}$, where $V_x$ and $V_y$ are the inputs. The magnitude of the average value of the multiplier output $V_0$ is \hspace{3cm}\rule{5cm}{0.4pt} V

\vspace{2cm}
 \includegraphics[width=0.8\linewidth]{Gate.png} 

\input{tables/table.tex}
Let the curret in load be I
\begin{center}
\begin{align}
     I &=\frac{V_s(peak)}{Z}\\
     &=\frac{230\sqrt{2}}{\frac{230}{\sqrt{2}}+j\frac{230}{\sqrt{2}}} \\
     &=\sqrt{2}(1-j)
\end{align}
\end{center}
voltage at attenuator 
\begin{align}
    V_1&=V_sA_1\\
    &=230\frac{1}{50\sqrt{2}}V\\
    &=\frac{4.6}{\sqrt{2}}V
\end{align}
\begin{center}
    \begin{align}
        V_y&=4.6\sin(\omega t+90)\\
        V_x&=2\sqrt{2}\sin(\omega t-45)\\
        V_0&=9.2\sqrt{2}(\frac{\cos(135)-\cos(2\omega t)}{2})\\
        &=4.6-4.6\sqrt{2}\cos(2\omega t)\\
        V_0<avg>&=4.6
    \end{align}
\end{center}
\end{document}